El desarrollo del presente informe se estructura en seis cap�tulos.

En el cap�tulo \ref{ref:chapter_antecedentes} se presentan antecedentes generales sobre el tema de la computaci�n distribu�da haciendo especial hincapi� en el an�lisis de dos soluciones existentes. Adem�s se describen los requerimientos y la factibilidad del sistema.

El cap�tulo \ref{ref:chapter_arquitectura} tiene como objetivo definir l�gicamente la soluci�n, estudiando aspectos como la arquitectura y funcionamiento del sistema.

En el cap�tulo \ref{ref:chapter_implementacion} se tratan los temas respectivos a la implementaci�n de la soluci�n, tales como las tecnolog�as a utilizar, la estructura interna de las piezas de software y metodolog�as utilizadas.

En el cap�tulo \ref{ref:chapter_comunicacion} se describen los aspectos relativos a la comunicaci�n de los diferentes elementos de software que componen el sistema. Se detallan los protocolos de comunicaci�n usados as� como tambi�n las soluciones a diversos problemas inherentes al uso de �stos.

En el cap�tulo \ref{ref:chapter_evaluacion} se presenta la aplicaci�n de prueba, donde se implementa el sistema para su evaluaci�n. Tambi�n son mostrados los resultados de las pruebas con sus respectivos an�lisis.

Finalmente, en el cap�tulo \ref{ref:chapter_conclusiones} se presentan las conclusiones extra�das del proceso de desarrollo del proyecto.

Adicionalmente hay una secci�n de anexos al final del documento, donde fueron relegados los desarrollos demasiado extensos o de naturaleza demasiado espec�fica que son referenciados a lo largo del documento.