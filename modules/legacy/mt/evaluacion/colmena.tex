Colmena es el nombre que se le ha dado al sistema de cluster virtual. El nombre sigue con la tradici�n de tomar un nombre que tenga un contexto natural, tradici�n com�n de aplicaciones de software desarrolladas para el laboratorio PROFC. Actualmente Colmena est� corriendo en la Universidad de Concepci�n usando el poder de c�mputo de diversos laboratorios de �sta.
\subsection{Estructura de la red de Colmena}
En la figura \ref{fig:image15} se muestra la estructura del cluster virtual que est� actualmente desplegado en la red de la Universidad de Concepci�n. El programa maestro se encuentra en uno de los servidores del PROFC. Los esclavos est�n distribuidos en diferentes grupos que comprenden los siguientes laboratorios:
\begin{figure}
\includegraphics[width=\textwidth]{images/image15.eps}
\caption{Estructura del cluster virtual Colmena de la Universidad de Concepci�n}
\label{fig:image15}
\end{figure}
\begin{itemize}
\item Laboratorio de redes del DIICC (10 computadores). 
\item Laboratorio PROFC (5 computadores).
\item Laboratorio de biolog�a marina (20 computadores).
\item Laboratorio de bot�nica (15 computadores). 
\end{itemize}
Adem�s el servidor del PROFC alberga un sitio web para Colmena\cite{colmenapage} (figura \ref{fig:image16}) desde el cu�l se puede acceder a las funciones del programa cliente, a excepci�n de las que se refieren a ingreso y manejo de proyectos, las cuales s�lo se permiten de manera interna por medio de la aplicaci�n de prueba descrita posteriormente en la secci�n \ref{ref:app_prueba}. La forma en que se instala el cluster virtual es detallada en el anexo \ref{anexo:instalacion_componentes}.
\begin{figure}
\includegraphics[width=\textwidth]{images/image16.eps}
\caption{P�gina principal del portal del cluster virtual de la Universidad de Concepci�n}
\label{fig:image16}
\end{figure}
\subsection{Comportamiento de programas esclavos}
Las diferentes pol�ticas de los laboratorios y formas de uso que los usuarios le dan a �stos repercuten en la disponibilidad de los esclavos para realizar c�mputos. La frecuencia de computadores conectados de los laboratorios de detallan en la tabla \ref{table:table01}.
\begin{table}
\centering
\begin{tabular}{ l | r | r | r | r || r || r}
 \hline			
   Laboratorio & Ma�ana & Mediod�a & Tarde & Noche & Promedio & Fin de semana\\
 \hline
   PROFC & 0 & 4 & 0& 3 & 1.8333 & Si\\
   DIICC Redes & 2 & 2 & 3 & 2 & 2.25& Si \\
   Biolog�a marina & 5 & 8 & 6 & 0 & 3& No \\
   Bot�nica & 3 & 5  & 4 & 0 & 1.916& No \\
   \hline  
	Total  & 10 & 19  & 13& 5 & 9 & \\ 
	\hline
\end{tabular}
\caption[Frecuencias de esclavos conectados por d�a]{Frecuencias de esclavos conectados por d�a. Las duraciones son: Ma�ana(8:00 a 12:00), Mediod�a (12:00 a 14:00), Tarde (14:00 a 20:00), Noche (20:00 a 8:00). El promedio de esclavos conectados por d�a de semana. El valor fin de semana indica si el laboratorio est� disponible los fines de semana. El promedio est� en \texttt{computadores disponibles por hora}.}
\label{table:table01}
\end{table}
El laboratorio PROFC en las horas laborales est� completamente en uso, ya que su personal se encuentra trabajando. En horas de almuerzo y en las noches es cuando el poder de c�mputo ocioso se encuentra disponible.

Los computadores del laboratorio DIICC Redes poseen dos sistemas operativos, Windows y Linux, donde actualmente el programa esclavo se encuentra corriendo �nicamente en Windows. Este laboratorio siempre se encuentra usado y la mayor�a del tiempo se encuentran en sistema Linux, lo que disminuye el aprovechamiento del poder ocioso. A pesar de esto, son los que entregan el mayor poder de c�mputo.

Los dos laboratorios de la facultad de Cs. Naturales entregan la mayor parte la mayor cantidad de computadores conectados simult�neamente en algunos periodos. Sin embargo, su promedio es bajo, ya que es pol�tica de este laboratorio apagarse en las noches y los fines de semanas.

Las mediciones de la tabla considera un d�a de semana. Si se hiciera la misma tabla considerando los fin de semanas, entonces se debe tener en cuenta que los laboratorios que no est�n disponibles en este per�odo bajar�n su disponibilidad de poder de c�mputo promedio.

Adicional a estas mediciones, se ha registrado un peak de 25 computadores simult�neamente disponibles. Esto ocurri� durante el mediod�a.
