El modo b�sico de uso de Q$^{2}$ADPZ es a trav�s de una aplicaci�n de l�nea de comando llamada \emph{cliente universal} y que le permite a los usuarios ingresar el programa a ejecutar. En este ingreso se puede especificar
\begin{itemize}
\item n�mero de ejecuciones del programa,
\item la ruta del ejecutable y los argumentos de l�nea de comando,
\item archivos de entrada y salida,
\item directorios donde residen los archivos,
\item utilidades a ser ejecutadas despu�s de las tareas individuales,
\item tiempo m�ximo asignado para la ejecuci�n de una tarea,
\item el orden en que los grupos de tareas ser�n ejecutados,
\item requerimientos de hardware (espacio en disco, memoria, velocidad y tipo de CPU) y software (sistema operativo y programas instalados).
\end{itemize}

Estos par�metros de configuraci�n son guardados en un archivo en formato XML. El ejecutable puede ser obtenido desde el disco local o descargado desde alguna direcci�n URL. Los archivos de entrada y salida son transferidos a los esclavos usando un servidor web dedicado a los datos.

Cada ejecuci�n corresponde a una tarea - la unidad de c�mputo m�s peque�a en Q$^{2}$ADPZ. Las tareas son agrupadas en trabajos - identificadas por un identificador y un nombre. El sistema permite la operaci�n de control sobre tareas, trabajos y usuarios. Usuarios avanzados pueden crear el archivo de configuraci�n manualmente o generarlo autom�ticamente.

Usuarios a�n m�s avanzados pueden prescindir del cliente universal y programar su propia \emph{aplicaci�n cliente}\footnote{user client application} que se comunique directamente con el maestro, permitiendo ingresar tareas din�micamente. Finalmente, los usuarios pueden programar sus propias \textit{bibliotecas esclavas}\footnote{user slave library} que en casos espec�ficos son, por lo general, m�s r�pidas que los programas ejecutables.

