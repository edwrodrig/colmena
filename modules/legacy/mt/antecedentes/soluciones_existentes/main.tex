Esta secci�n tiene como objetivo mostrar soluciones existentes que permitan caracterizar lo que actualmente existe en el �mbito de la computaci�n distribuida. Existen muchos proyectos que tienen un objetivo similar al del presente proyecto. Por esta raz�n ser�a demasiado extenso un an�lisis de todos ellos y as� se han elegido dos soluciones a analizar. \emph{BOINC}\cite{boinc}, uno de los proyectos m�s populares en lo que se refiere a computaci�n p�blica global, y \emph{Q$^{2}$ADPZ}\cite{q2adpz}, la soluci�n m�s cercana en cuanto a objetivos y contexto a la presentada en este documento. Adem�s se contrasta el cluster virtual contra la soluci�n ofrecida por la computaci�n de nube.
\subsection{Boinc}\label{ref:solucionesexistentes_boinc}
\emph{BOINC} (\textbf{B}erkeley \textbf{O}pen \textbf{I}nfrastructure for \textbf{N}etwork \textbf{C}omputing) es un popular sistema de software que permite a cient�ficos e investigadores crear y operar proyectos de recursos computacionales p�blicos que facilitan muchos voluntarios alrededor del mundo. Entre los proyectos m�s conocidos se encuentran \emph{SETI@Home}\cite{setiathomepage}, un proyecto que tiene como objetivo la busqueda de inteligencia extraterrestre, y \emph{MilkyWay@home}\cite{milkywayathomepage} donde se usa el poder de BOINC para crear un mapa tridimiensional certero de la \emph{V�a L�ctea}. Actualmente\footnote{21 de Junio de 2009}, la red mundial de BOINC tiene alrededor de 570.000 computadores que se traduce en un poder de c�mputo promedio de unos 2.682 TeraFLOPs. Alternativamente BOINC puede ser usado por peque�as redes institucionales involucrando sus propios computadores.
\subsubsection{Caracter�sticas}
BOINC fue concebido bajo el paradigma de la \emph{computaci�n p�blica}\footnote{Tambi�n es conocida como \emph{computaci�n global} o \emph{computaci�n p2p}} donde los recursos de personas particulares son dispuestos de manera p�blica para hacer supercomputaci�n de alto rendimiento.

BOINC establece una relaci�n asim�trica entre proyectos y participantes.
Los proyectos t�picamente son peque�os grupos de investigaci�n acad�mica con limitado poder de c�mputo y son los usuarios finales de BOINC. Los participances son individuos que poseen computadores con Windows, Linux u otro sistema operativo conectados a Internet.

Para incentivar la participaci�n, BOINC posee un elaborado sistema de incentivos en la forma de cr�ditos que premian a los usuarios seg�n cuanto hayan contribuido en tiempo de procesamiento. El sistema de incentivos tambi�n tiene la cualidad de ser altamente resistente a que los usuarios intenten \emph{trampas} para ganar cr�ditos de forma indebida.
\subsubsection{Arquitectura y funcionamiento}
Un proyecto de BOINC corresponde a una organizaci�n o grupo de investigaci�n que quiere disponer del poder de c�mputo de los voluntarios para sus fines. Estos proyectos son identificados por una URL que corresponde a la p�gina principal de su sitio y sirve como directorio de su servidor. Los participantes se registran a proyectos. Un proyecto puede involucrar a una o m�s aplicaciones, las que pueden cambiar con el tiempo. En la figura \ref{fig:image05} se muestra el esquema antes descrito.

La complejidad del servidor de un proyecto BOINC est� centrada alrededor de una base de datos relacional que almacena la descripci�n de aplicaciones, plataformas, versiones, resultados, cuentas, equipos, entre otros. Las funciones del servidor son efectuadas por un conjunto de servicios web y procesos demonios. Los servidores de datos manipulan las subidas de archivos usando un mecanismo de certificados que asegura que s�lo los archivos legitimados con tama�os de archivos prescritos pueden ser subidos.
\begin{figure}
\includegraphics[width=\textwidth]{images/image05.eps}
\caption{Esquema de los participantes de BOINC}
\label{fig:image05}
\end{figure}
La descarga de archivos es llevada a cabo mediante HTTP plano. BOINC provee herramientas para crear, empezar, detener, consultar proyectos; agregar nuevas aplicaciones, plataformas, versiones de aplicaciones, y monitorear el rendimiento del servidor.

Los participantes se pueden unir a un proyecto de BOINC visitando la p�gina del proyecto, llenando la solicitud de registro y descargando el cliente BOINC. El cliente puede operar en muchos modos: como un salvapantallas que muestra gr�ficos correspondientes a la aplicaci�n que se encuentra corriendo; como un servicio de Windows que corre incluso cuando el usuario no ha ingresado; como una aplicaci�n que provee un despliegue tabular de los proyectos, trabajos y transferencias de archivos, y como un programa de l�nea de comando que se comunica mediante la entrada y salida est�ndar.
\subsubsection{Desventajas}
BOINC es un sistema demasiado general, haci�ndolo el t�pico caso de \emph{aplastar una hormiga con una aplanadora}. Si bien puede ser adaptado a muchas situaciones, esto no significa que sea la mejor opci�n para el caso particular del presente proyecto ya que se necesita de consideraciones extra que resultan innecesarias y s�lo complican la implementaci�n. En espec�fico se pueden distinguir los siguientes aspectos:
\begin{itemize}
\item BOINC no permite una implementaci�n minimalista. Para poder utilizar el sistema se necesita de la creaci�n de un proyecto, el que debe ser implementado cumpliendo las reglas establecidas para un servidor BOINC. Seg�n la documentaci�n de BOINC, se asegura que en poco tiempo y sin un gran esfuerzo en implementaci�n se puede incorporar el sistema a una aplicaci�n existente, pero el hecho de pr�cticamente necesitar elaborar un sitio web, montar una base de datos y adem�s ce�irse al modelo BOINC son cosas que contradicen lo anteriormente mencionado.
\item El sistema de recompensas a voluntarios es bueno cuando se necesita incentivar a los usuarios a participar en los proyectos para lograr que pongan a su disposici�n sus computadores. Sin embargo, cuando una instituci�n quiere usar sus computadores ociosos, todo este sistema de recompensas es innecesario y carece de sentido. Si consideramos que para poder implementar un proyecto debemos considerar cuentas de usuarios entonces est� caracter�stica se vuelve un obst�culo que dificulta el f�cil uso del sistema.
\end{itemize}


			
\subsection{Q$^{2}$ADPZ}\label{ref:solucionesexistentes_q2adpz}
\emph{BOINC} (\textbf{B}erkeley \textbf{O}pen \textbf{I}nfrastructure for \textbf{N}etwork \textbf{C}omputing) es un popular sistema de software que permite a cient�ficos e investigadores crear y operar proyectos de recursos computacionales p�blicos que facilitan muchos voluntarios alrededor del mundo. Entre los proyectos m�s conocidos se encuentran \emph{SETI@Home}\cite{setiathomepage}, un proyecto que tiene como objetivo la busqueda de inteligencia extraterrestre, y \emph{MilkyWay@home}\cite{milkywayathomepage} donde se usa el poder de BOINC para crear un mapa tridimiensional certero de la \emph{V�a L�ctea}. Actualmente\footnote{21 de Junio de 2009}, la red mundial de BOINC tiene alrededor de 570.000 computadores que se traduce en un poder de c�mputo promedio de unos 2.682 TeraFLOPs. Alternativamente BOINC puede ser usado por peque�as redes institucionales involucrando sus propios computadores.
\subsubsection{Caracter�sticas}
BOINC fue concebido bajo el paradigma de la \emph{computaci�n p�blica}\footnote{Tambi�n es conocida como \emph{computaci�n global} o \emph{computaci�n p2p}} donde los recursos de personas particulares son dispuestos de manera p�blica para hacer supercomputaci�n de alto rendimiento.

BOINC establece una relaci�n asim�trica entre proyectos y participantes.
Los proyectos t�picamente son peque�os grupos de investigaci�n acad�mica con limitado poder de c�mputo y son los usuarios finales de BOINC. Los participances son individuos que poseen computadores con Windows, Linux u otro sistema operativo conectados a Internet.

Para incentivar la participaci�n, BOINC posee un elaborado sistema de incentivos en la forma de cr�ditos que premian a los usuarios seg�n cuanto hayan contribuido en tiempo de procesamiento. El sistema de incentivos tambi�n tiene la cualidad de ser altamente resistente a que los usuarios intenten \emph{trampas} para ganar cr�ditos de forma indebida.
\subsubsection{Arquitectura y funcionamiento}
Un proyecto de BOINC corresponde a una organizaci�n o grupo de investigaci�n que quiere disponer del poder de c�mputo de los voluntarios para sus fines. Estos proyectos son identificados por una URL que corresponde a la p�gina principal de su sitio y sirve como directorio de su servidor. Los participantes se registran a proyectos. Un proyecto puede involucrar a una o m�s aplicaciones, las que pueden cambiar con el tiempo. En la figura \ref{fig:image05} se muestra el esquema antes descrito.

La complejidad del servidor de un proyecto BOINC est� centrada alrededor de una base de datos relacional que almacena la descripci�n de aplicaciones, plataformas, versiones, resultados, cuentas, equipos, entre otros. Las funciones del servidor son efectuadas por un conjunto de servicios web y procesos demonios. Los servidores de datos manipulan las subidas de archivos usando un mecanismo de certificados que asegura que s�lo los archivos legitimados con tama�os de archivos prescritos pueden ser subidos.
\begin{figure}
\includegraphics[width=\textwidth]{images/image05.eps}
\caption{Esquema de los participantes de BOINC}
\label{fig:image05}
\end{figure}
La descarga de archivos es llevada a cabo mediante HTTP plano. BOINC provee herramientas para crear, empezar, detener, consultar proyectos; agregar nuevas aplicaciones, plataformas, versiones de aplicaciones, y monitorear el rendimiento del servidor.

Los participantes se pueden unir a un proyecto de BOINC visitando la p�gina del proyecto, llenando la solicitud de registro y descargando el cliente BOINC. El cliente puede operar en muchos modos: como un salvapantallas que muestra gr�ficos correspondientes a la aplicaci�n que se encuentra corriendo; como un servicio de Windows que corre incluso cuando el usuario no ha ingresado; como una aplicaci�n que provee un despliegue tabular de los proyectos, trabajos y transferencias de archivos, y como un programa de l�nea de comando que se comunica mediante la entrada y salida est�ndar.
\subsubsection{Desventajas}
BOINC es un sistema demasiado general, haci�ndolo el t�pico caso de \emph{aplastar una hormiga con una aplanadora}. Si bien puede ser adaptado a muchas situaciones, esto no significa que sea la mejor opci�n para el caso particular del presente proyecto ya que se necesita de consideraciones extra que resultan innecesarias y s�lo complican la implementaci�n. En espec�fico se pueden distinguir los siguientes aspectos:
\begin{itemize}
\item BOINC no permite una implementaci�n minimalista. Para poder utilizar el sistema se necesita de la creaci�n de un proyecto, el que debe ser implementado cumpliendo las reglas establecidas para un servidor BOINC. Seg�n la documentaci�n de BOINC, se asegura que en poco tiempo y sin un gran esfuerzo en implementaci�n se puede incorporar el sistema a una aplicaci�n existente, pero el hecho de pr�cticamente necesitar elaborar un sitio web, montar una base de datos y adem�s ce�irse al modelo BOINC son cosas que contradicen lo anteriormente mencionado.
\item El sistema de recompensas a voluntarios es bueno cuando se necesita incentivar a los usuarios a participar en los proyectos para lograr que pongan a su disposici�n sus computadores. Sin embargo, cuando una instituci�n quiere usar sus computadores ociosos, todo este sistema de recompensas es innecesario y carece de sentido. Si consideramos que para poder implementar un proyecto debemos considerar cuentas de usuarios entonces est� caracter�stica se vuelve un obst�culo que dificulta el f�cil uso del sistema.
\end{itemize}



\subsection{Computaci�n de nube}
Como se mencion� anteriormente, cuando se habla de computaci�n de nube no se est� hablando de una tecnolog�a en particular, sino de una soluci�n ofrecida como un servicio. Elegir entre pagar por un servicio o implantar tecnolog�a se puede traducir a la pregunta : \emph{�Es preferible cultivar mi comida o comprarla en el supermercado?}. Dentro del contexto del proyecto esta pregunta ser�a : \emph{�Es preferible usar el cluster virtual o simplemente pagar por el servicio de c�mputo?}.
Para responder a esta pregunta hay que remitirse al objetivo principal del proyecto que es \emph{''aprovechar el poder de c�mputo de los computadores ociosos''}. De esta sentencia se deduce que el cluster virtual est� pensado para aprovechar un potencial no utilizado. Si existe tal potencial entonces �por qu� se tendr�a que pagar adicionalmente por �l?. En resumen, la computaci�n de nube resuelve el problema de necesidad de poder de c�mputo mediante recursos econ�micos, en cambio el sistema de este proyecto resuelve este problema mediante el uso de recursos subutizados, lo que significa una soluci�n m�s �ptima.

