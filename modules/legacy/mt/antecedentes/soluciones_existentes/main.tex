Esta secci�n tiene como objetivo mostrar soluciones existentes que permitan caracterizar lo que actualmente existe en el �mbito de la computaci�n distribuida. Existen muchos proyectos que tienen un objetivo similar al del presente proyecto. Por esta raz�n ser�a demasiado extenso un an�lisis de todos ellos y as� se han elegido dos soluciones a analizar. \emph{BOINC}\cite{boinc}, uno de los proyectos m�s populares en lo que se refiere a computaci�n p�blica global, y \emph{Q$^{2}$ADPZ}\cite{q2adpz}, la soluci�n m�s cercana en cuanto a objetivos y contexto a la presentada en este documento. Adem�s se contrasta el cluster virtual contra la soluci�n ofrecida por la computaci�n de nube.
\subsection{Boinc}\label{ref:solucionesexistentes_boinc}
\chapter{Instalaci�n y Configuraci�n de los Componentes}\label{anexo:instalacion_componentes}
\section{Instalaci�n del programa maestro}
Para instalar el programa maestro en un computador simplemente se necesita copiar el ejecutable\footnote{\texttt{master.exe} en Windows, \texttt{master} en otros sistemas operativos} en una carpeta y tener configuradas las bibliotecas de tiempo de ejecuci�n de Qt4 en el sistema operativo. Estas bibliotecas se encuentran disponibles en la p�gina \url{http://www.qtsoftware.com/downloads}. En los sistemas operativos basados en Windows NT, para tener las librerias disponibles simplemente se necesitan copiar los dll correspondientes en la misma carpeta en la que se encuentra el ejecutable.
\section{Configuraci�n del programa maestro}
Para ver todas las opciones de que tiene el programa maestro hay que ejecutar en la l�nea de comando \texttt{master help}, entonces aparecer� la siguiente salida.
\begin{verbatim}
master                        Run VC Master
master [help]                 Show Help
master info                   Show configuration info
master setslaveport port      Set master slave listen port
master setclientport port     Set master client listen port
master setslavetimeout msecs  Set master slave delete timeout time
master setrootpassword pass   Set master root password
\end{verbatim}
\subsection{Asignaci�n del puerto de comunicaci�n de esclavos}
El puerto de comunicaci�n de esclavos, es el puerto UDP por el cual el programa maestro recibe los mensajes de los esclavos. Si se posee un cortafuegos, hay que cerciorarse que el computador que correr� el programa maestro tenga tal puerto abierto. Por defecto, el programa maestro tiene asignado el puerto 40000 para tal fin. Si se desea cambiar el puerto por defecto se puede hacer con el comando \texttt{master setslaveport port} donde \texttt{port} es el nuevo puerto a asignar.
\begin{ejemplo} Si se desea asignar el puerto 4000 como puerto de comunicaci�n de esclavos entonces se debe ingresar por la l�nea de comando \texttt{master setslaveport 4000}.
\end{ejemplo}
\subsection{Asignaci�n del puerto de comunicaci�n de clientes}
El puerto de comunicaci�n de clientes, es el puerto TCP por el cual el programa maestro se comunica con los clientes. Si se posee un cortafuegos, hay que cerciorarse que el computador que correr� el programa maestro tenga tal puerto abierto. Por defecto, el programa maestro tiene asignado el puerto 50000 para tal fin. Si se desea cambiar el puerto por defecto se puede hacer con el comando \texttt{master setclientport port} donde \texttt{port} es el nuevo puerto a asignar.
\begin{ejemplo} Si se desea asignar el puerto 5000 como puerto de comunicaci�n de clientes entonces se debe ingresar por l�nea de comando \texttt{master setslaveport 5000}.
\end{ejemplo}
\subsection{Configuraci�n del tiempo de espera de esclavos}
El tiempo de espera de esclavos es el lapso de tiempo de espera desde el �ltimo mensaje recibido de un esclavo donde si no se ha recibido un nuevo mensaje, entonces el esclavo es considerado por el maestro como desconectado y es eliminado de su lista de esclavos disponibles. Por defecto, el tiempo de espera de esclavo es de 30000 milisegundos. Si se desea cambiar el tiempo de espera se puede hacer con el comando \texttt{master setslavetimeout msecs} donde \texttt{msecs} es el tiempo de espera en milisegundos a asignar.
\begin{ejemplo} Si se desea configurar el tiempo de espera de esclavos en 1 minuto entonces se debe ingresar por l�nea de comando \texttt{master setslavetimeout 60000}.
\end{ejemplo}
\subsection{Asignaci�n de la contrase�a de administrador}
El programa maestro posee una cuenta de administrador por defecto de usuario \texttt{root} y contrase�a \texttt{pass}. Con esta cuenta de usuario se tiene control total sobre el sistema. Si se desea cambiar la contrase�a de administrador se puede hacer con el comando \texttt{master setrootpassword pass} donde \texttt{pass} es la contrase�a nueva.
\begin{ejemplo} Si se desea asignar \texttt{root123} como contrase�a entonces se debe ingresar por l�nea de comando \texttt{master setrootpassword root123}.
\end{ejemplo}
\subsection{Ver configuraci�n actual del sistema}
Con esta opci�n se puede ver los valores de la configuraci�n del maestro vigentes. Para verlos simplemente hay que ingresar en el comando \texttt{master info}. Si la configuraci�n del maestro solo tiene valores por defecto, entonces la salida mostrada deber�a ser:
\begin{verbatim}
Slave Listen Port  : 40000
Client Listen Port : 50000
Slave Timeout      : 30000
Root Password      : pass
\end{verbatim}
Si se hubieran reasignado los valores de la configuraci�n con los establecidos en los ejemplos descritos anteriormente, entonces la salida ser�a:
\begin{verbatim}
Slave Listen Port  : 4000
Client Listen Port : 5000
Slave Timeout      : 60000
Root Password      : root123
\end{verbatim}
\section{Ejecuci�n del programa maestro}
Para correr el programa maestro simplemente hay que ingresar el comando \texttt{master} para levantar el cluster virtual. El programa maestro es un proceso permanente, por eso se recomienda ejecutarlo como un proceso de fondo. Al ejecutar el programa maestro aparecer� la siguiente salida:
\begin{verbatim}
======================================
        Virtual Cluster System
    by Edwin Rodriguez Leon (2009)
======================================
           Master App v0.01
======================================
Loading configuration...
Configuration loaded.
Creating Database...
Database opened.
Init server...
Server running at port 40000 .
Listening clients at port 50000 .
\end{verbatim}
En este punto, el cluster virtual ya se encuentra funcionando, pero sin ning�n esclavo al que se le puedan enviar tareas para su procesamiento. Sin embargo esta listo para recibir tareas y agregar esclavos al sistema.
\section{Instalaci�n del programa esclavo}
La instalaci�n b�sica del programa esclavo es similar al del programa maestro es en un computador simplemente se necesita copiar el ejecutable\footnote{\texttt{colmena.exe} en Windows, \texttt{colmena} en Linux} en la carpeta de Colmena\footnote{\texttt{C:{\textbackslash}colmena} en Windows, \texttt{/usr/local/colmena} en Linux} y tener configuradas las bibliotecas de tiempo de ejecuci�n de Qt4 en el sistema operativo.
\subsection{Agregar aplicaciones al programa esclavo}
Para agregar aplicaciones al sistema simplemente se deben copiar los archivos en la carpeta \texttt{apps} dentro de la carpeta principal de Colmena. Si las aplicaciones necesitan la configuraci�n de variables de entornos entonces se pueden definir agregando un archivo de texto en la carpeta \texttt{envvars}. Este archivo de texto debe tener el siguiente formato.
\begin{verbatim}
NOMBRE_VARIABLE_ENTORNO_1
VALOR_VARIABLE_ENTORNO_1
NOMBRE_VARIABLE_ENTORNO_2
VALOR_VARIABLE_ENTORNO_2
...
NOMBRE_VARIABLE_ENTORNO_n
VALOR_VARIABLE_ENTORNO_n
\end{verbatim}
No hay diferencia en agregar una variable de entorno a un archivo existente o crear un nuevo archivo donde se defina.
\subsection{Instalaci�n para modo salvapantallas}
Si se desea que el esclavo funciona solamente cuando est� en modo salvapantallas, entonces adicionalmente a lo descrito en los pasos anteriores se deben copiar los archivos correspondientes al salvapantallas\footnote{\texttt{sscolmena.scr}} y al programa de seteo de prioridad de procesos\footnote{\texttt{SetPriority.exe}}. El archivo \texttt{sscolmena.scr} debe ser instalado en el sistema operativo\footnote{apretar con el bot�n derecho el archivo y seleccionar la opci�n \emph{Instalar}}. 
\section{Configuraci�n del programa esclavo}
Para ver todas las opciones de que tiene el programa esclavo hay que ejecutar en la l�nea de comando \texttt{colmena help}, entonces aparecer� la siguiente salida.
\begin{verbatim}
Usage :
run                    Run VC Slave
[help]                 Show Help
info                   Show configuration info
setmasterip ip         Set slave master ip
setmasterport port     Set slave master port
setlistenport port     Set slave listen port
setmaxtasks tasks      Set slave max tasks
setpriority priority   Set if slave is always on high priority
setgroup group         Set slave group
setssaverport port     Set ssaver port
\end{verbatim}
\subsection{Asignaci�n de la direcci�n IP del maestro}
La direcci�n IP del maestro, es la direcci�n IP del computador que est� corriento el programa maestro. Por defecto, el programa esclavo tiene asignada la direcci�n \texttt{127.0.0.1 (localhost)} para tal fin la cual no corresponde en la mayor�a de los casos a menos que el programa maestro este corriendo en la misma m�quina que el esclavo. Si se desea cambiar el puerto por defecto se puede hacer con el comando \texttt{master setmasterip ip} donde \texttt{ip} es la nueva direcci�n a asignar.
\begin{ejemplo} Si se desea asignar la ip 152.1.2.3 como direcci�n IP del maestro entonces se debe ingresar por la l�nea de comando \texttt{colmena setmasterip 152.1.2.3}.
\end{ejemplo}
\subsection{Asignaci�n del puerto de comunicaci�n del maestro}
El puerto de comunicaci�n del maestro, es el puerto UDP por el cual el programa maestro establece la comunicaci�n con los esclavos. Por defecto, el programa esclavo tiene asignado el puerto 40000. Si se desea cambiar el puerto por defecto se puede hacer con el comando \texttt{colmena setmasterport port} donde \texttt{port} es el nuevo puerto a asignar.
\begin{ejemplo} Si se desea asignar el puerto 4000 como puerto de comunicaci�n del maestro entonces se debe ingresar por l�nea de comando \texttt{colmena setmasterport 4000}.
\end{ejemplo}
\subsection{Asignaci�n del puerto de comunicaci�n del esclavo}
El puerto de comunicaci�n del maestro es el puerto UDP por el cual el programa esclavo se comunica con el maestro. Si se posee un cortafuegos, hay que cerciorarse que el computador que correr� el programa esclavo tenga tal puerto abierto. Por defecto, el programa esclavo tiene asignado el puerto 40001. Si se desea cambiar el puerto por defecto se puede hacer con el comando \texttt{colmena setlistenport port} donde \texttt{port} es el nuevo puerto a asignar.
\begin{ejemplo} Si se desea asignar el puerto 4001 como puerto de comunicaci�n del maestro entonces se debe ingresar por l�nea de comando \texttt{colmena setlistenport 4001}.
\end{ejemplo}
\subsection{Configuraci�n del n�mero m�ximo de tareas simultaneas}
Como su nombre lo dice, el n�mero m�ximo de tareas simultaneas corresponde al l�mite de tareas que el esclavo puede ejecutar a la vez. Este valor es particularmente �til cuando se dispone de un computador con muchos procesadores disponibles. Por defecto, el n�mero m�ximo de tareas simultaneas es 1. Si se desea cambiar este valor se puede hacer con el comando \texttt{colmena setmaxtasks tasks} donde \texttt{tasks} es el nuevo m�ximo a asignar.
\begin{ejemplo} Si se desea configurar el n�mero m�ximo de tareas simultaneas en 2 tareas simultaneas entonces se debe ingresar por linea de comando \texttt{colmena setmaxtasks 2}.
\end{ejemplo}
\subsection{Asignaci�n de la prioridad del esclavo}
La prioridad del esclavo es un valor booleano que establece si el seclavo estar� en modo dedicado o no.Si el valor es \texttt{0} entonces el esclavo solo recibir� tareas cuando este en modo salvapantallas. Si el valor es \texttt{1} entonces se asume que el esclavo esta en modo dedicado y recibir� tareas siempre. Por defecto, el esclavo esta configurado para que funcione en modo no dedicado. Si se desea cambiar la prioridad del esclavo entonces se puede hacer con el comando \texttt{colmena setpriority priority} donde \texttt{priority} es el valor booleano correspondiente
\begin{ejemplo} Si se desea asignar cambiar a modo dedicado entonces se debe ingresar por l�nea de comando \texttt{colmena setpriority 1}.
\end{ejemplo}
\subsection{Asignaci�n del grupo del esclavo}
Los esclavos son asociados en grupos que determinan las p�ginas que van a desplegar en sus salvapantallas. El grupo de un esclavo es el nombre del grupo a que pertenece es esclavo. Por defecto, el esclavo esta configurado sin ning�n grupo. Si se desea cambiar la grupo del esclavo entonces se puede hacer con el comando \texttt{colmena setgroup group} donde \texttt{group} es el nombre del grupo.
\begin{ejemplo} Si se desea asignar el grupo \texttt{profc} entonces se debe ingresar por l�nea de comando \texttt{colmena setgroup profc}.
\end{ejemplo}
\subsection{Asignaci�n del puerto de comunicaci�n del salvapantallas}
El puerto de comunicaci�n del salvapantallas es el puerto TCP por el cual el esclavo y el salvapantallas se comunican. Por defecto, el esclavo esta configurado con el puerto 18765. Si se desea cambiar el puerto de comunicaci�n del salvapantallas se puede hacer con el comando \texttt{colmena setssaverport port} donde \texttt{port} es el nuevo puerto a asignar.
\begin{ejemplo} Si se desea asignar el puerto 1876 como puerto de comunicaci�n del salvapantallas entonces se debe ingresar por l�nea de comando \texttt{colmena setssaverport 1876}
\end{ejemplo}
\subsection{Ver configuraci�n actual del esclavo}
Con esta opci�n se puede ver los valores de la configuraci�n del esclavo vigentes. Para verlos simplemente hay que ingresar en el comando \texttt{colmena info}. Si la configuraci�n del esclavo solo tiene valores por defecto, entonces la salida mostrada deber�a ser:
\begin{verbatim}
Version                 : 0.02
Master Ip               : 127.0.0.1
Master Port             : 40000
Listen Port             : 40001
Max Tasks               : 1
Always on high priority : 0
Group                   : 
Screensaver Port        : 18765
\end{verbatim}
Si se hubieran reasignado los valores de la configuraci�n con los establecidos en los ejemplos descritos anteriormente, entonces la salida ser�a:
\begin{verbatim}
Version                 : 0.02
Master Ip               : 152.1.2.3
Master Port             : 4000
Listen Port             : 4001
Max Tasks               : 2
Always on high priority : 1
Group                   : profc
Screensaver Port        : 1876
\end{verbatim}
\section{Ejecuci�n del programa esclavo}
Para correr el programa esclavo simplemente hay que ingresar el comando \texttt{colmena run}. El programa esclavo es un proceso permanente, por eso se recomienda ejecutarlo como un proceso de fondo. El programa esclavo se demora un poco en estar disponible para recibir tareas debido a las rutinas de reconocimiento y verificaci�n de aplicaciones disponibles. Cuando el esclavo se encuentre disponible en la salida del maestro aparecer� un mensaje notificando este hecho :
\begin{verbatim}
New slave 1 arrived. Starting registering protocol.
\end{verbatim}
En este punto, el esclavo esta disponible para que el maestro le env�e tareas a ejecutar, en otras palabras, se ha agregado un nodo de c�mputo adicional al cluster virtual.
\section{Instalador autom�tico del esclavo para Windows XP}
Por conveniencia, se cre� un programa instalador autom�tico para Windows XP con InnoSetup\cite{innosetuppage} que tiene el fin de facilitar el proceso de configuraci�n del esclavo. Este instalador deja el programa esclavo como un servicio de Windows permanente que esta preconfigurado para que se conecte a un maestro en espec�fico.
\section{Instalador de paquetes de aplicaciones para Windows XP}
Tambi�n existe un instalador de aplicaciones para el esclavo. Corriendo este instalador agrega autom�ticamente aplicaciones disponibles por el esclavo.
\chapter{Descripci�n de Mensajes y Paquetes de Comunicaci�n}\label{anexo:definicion_mensaje}
Los mensajes o paquetes de comunicaci�n son descritos de la siguiente forma:
\begin{center}
\texttt{| tipo\_1 nombre\_1 | tipo\_2 nombre\_2 | ... | tipo\_n nombre\_n |\\origen>destino}
\end{center}
donde la l�nea superior separa los campos del mensaje con \texttt{|}. Cada campo tiene dos palabras donde la primera describe el tipo de dato que tiene y el segundo es un nombre identificador. En la segunda l�nea se describe la direcci�n en que va el mensaje. Tanto \texttt{origen} como \texttt{destino} son identificadores de entidades. El sentido del caracter \texttt{>} determina cual es la orientaci�n de la comunicaci�n. Eventualmente las dos l�neas pueden encontrarse concatenadas en una sola por conveniencia.
			
\subsection{Q$^{2}$ADPZ}\label{ref:solucionesexistentes_q2adpz}
\chapter{Instalaci�n y Configuraci�n de los Componentes}\label{anexo:instalacion_componentes}
\section{Instalaci�n del programa maestro}
Para instalar el programa maestro en un computador simplemente se necesita copiar el ejecutable\footnote{\texttt{master.exe} en Windows, \texttt{master} en otros sistemas operativos} en una carpeta y tener configuradas las bibliotecas de tiempo de ejecuci�n de Qt4 en el sistema operativo. Estas bibliotecas se encuentran disponibles en la p�gina \url{http://www.qtsoftware.com/downloads}. En los sistemas operativos basados en Windows NT, para tener las librerias disponibles simplemente se necesitan copiar los dll correspondientes en la misma carpeta en la que se encuentra el ejecutable.
\section{Configuraci�n del programa maestro}
Para ver todas las opciones de que tiene el programa maestro hay que ejecutar en la l�nea de comando \texttt{master help}, entonces aparecer� la siguiente salida.
\begin{verbatim}
master                        Run VC Master
master [help]                 Show Help
master info                   Show configuration info
master setslaveport port      Set master slave listen port
master setclientport port     Set master client listen port
master setslavetimeout msecs  Set master slave delete timeout time
master setrootpassword pass   Set master root password
\end{verbatim}
\subsection{Asignaci�n del puerto de comunicaci�n de esclavos}
El puerto de comunicaci�n de esclavos, es el puerto UDP por el cual el programa maestro recibe los mensajes de los esclavos. Si se posee un cortafuegos, hay que cerciorarse que el computador que correr� el programa maestro tenga tal puerto abierto. Por defecto, el programa maestro tiene asignado el puerto 40000 para tal fin. Si se desea cambiar el puerto por defecto se puede hacer con el comando \texttt{master setslaveport port} donde \texttt{port} es el nuevo puerto a asignar.
\begin{ejemplo} Si se desea asignar el puerto 4000 como puerto de comunicaci�n de esclavos entonces se debe ingresar por la l�nea de comando \texttt{master setslaveport 4000}.
\end{ejemplo}
\subsection{Asignaci�n del puerto de comunicaci�n de clientes}
El puerto de comunicaci�n de clientes, es el puerto TCP por el cual el programa maestro se comunica con los clientes. Si se posee un cortafuegos, hay que cerciorarse que el computador que correr� el programa maestro tenga tal puerto abierto. Por defecto, el programa maestro tiene asignado el puerto 50000 para tal fin. Si se desea cambiar el puerto por defecto se puede hacer con el comando \texttt{master setclientport port} donde \texttt{port} es el nuevo puerto a asignar.
\begin{ejemplo} Si se desea asignar el puerto 5000 como puerto de comunicaci�n de clientes entonces se debe ingresar por l�nea de comando \texttt{master setslaveport 5000}.
\end{ejemplo}
\subsection{Configuraci�n del tiempo de espera de esclavos}
El tiempo de espera de esclavos es el lapso de tiempo de espera desde el �ltimo mensaje recibido de un esclavo donde si no se ha recibido un nuevo mensaje, entonces el esclavo es considerado por el maestro como desconectado y es eliminado de su lista de esclavos disponibles. Por defecto, el tiempo de espera de esclavo es de 30000 milisegundos. Si se desea cambiar el tiempo de espera se puede hacer con el comando \texttt{master setslavetimeout msecs} donde \texttt{msecs} es el tiempo de espera en milisegundos a asignar.
\begin{ejemplo} Si se desea configurar el tiempo de espera de esclavos en 1 minuto entonces se debe ingresar por l�nea de comando \texttt{master setslavetimeout 60000}.
\end{ejemplo}
\subsection{Asignaci�n de la contrase�a de administrador}
El programa maestro posee una cuenta de administrador por defecto de usuario \texttt{root} y contrase�a \texttt{pass}. Con esta cuenta de usuario se tiene control total sobre el sistema. Si se desea cambiar la contrase�a de administrador se puede hacer con el comando \texttt{master setrootpassword pass} donde \texttt{pass} es la contrase�a nueva.
\begin{ejemplo} Si se desea asignar \texttt{root123} como contrase�a entonces se debe ingresar por l�nea de comando \texttt{master setrootpassword root123}.
\end{ejemplo}
\subsection{Ver configuraci�n actual del sistema}
Con esta opci�n se puede ver los valores de la configuraci�n del maestro vigentes. Para verlos simplemente hay que ingresar en el comando \texttt{master info}. Si la configuraci�n del maestro solo tiene valores por defecto, entonces la salida mostrada deber�a ser:
\begin{verbatim}
Slave Listen Port  : 40000
Client Listen Port : 50000
Slave Timeout      : 30000
Root Password      : pass
\end{verbatim}
Si se hubieran reasignado los valores de la configuraci�n con los establecidos en los ejemplos descritos anteriormente, entonces la salida ser�a:
\begin{verbatim}
Slave Listen Port  : 4000
Client Listen Port : 5000
Slave Timeout      : 60000
Root Password      : root123
\end{verbatim}
\section{Ejecuci�n del programa maestro}
Para correr el programa maestro simplemente hay que ingresar el comando \texttt{master} para levantar el cluster virtual. El programa maestro es un proceso permanente, por eso se recomienda ejecutarlo como un proceso de fondo. Al ejecutar el programa maestro aparecer� la siguiente salida:
\begin{verbatim}
======================================
        Virtual Cluster System
    by Edwin Rodriguez Leon (2009)
======================================
           Master App v0.01
======================================
Loading configuration...
Configuration loaded.
Creating Database...
Database opened.
Init server...
Server running at port 40000 .
Listening clients at port 50000 .
\end{verbatim}
En este punto, el cluster virtual ya se encuentra funcionando, pero sin ning�n esclavo al que se le puedan enviar tareas para su procesamiento. Sin embargo esta listo para recibir tareas y agregar esclavos al sistema.
\section{Instalaci�n del programa esclavo}
La instalaci�n b�sica del programa esclavo es similar al del programa maestro es en un computador simplemente se necesita copiar el ejecutable\footnote{\texttt{colmena.exe} en Windows, \texttt{colmena} en Linux} en la carpeta de Colmena\footnote{\texttt{C:{\textbackslash}colmena} en Windows, \texttt{/usr/local/colmena} en Linux} y tener configuradas las bibliotecas de tiempo de ejecuci�n de Qt4 en el sistema operativo.
\subsection{Agregar aplicaciones al programa esclavo}
Para agregar aplicaciones al sistema simplemente se deben copiar los archivos en la carpeta \texttt{apps} dentro de la carpeta principal de Colmena. Si las aplicaciones necesitan la configuraci�n de variables de entornos entonces se pueden definir agregando un archivo de texto en la carpeta \texttt{envvars}. Este archivo de texto debe tener el siguiente formato.
\begin{verbatim}
NOMBRE_VARIABLE_ENTORNO_1
VALOR_VARIABLE_ENTORNO_1
NOMBRE_VARIABLE_ENTORNO_2
VALOR_VARIABLE_ENTORNO_2
...
NOMBRE_VARIABLE_ENTORNO_n
VALOR_VARIABLE_ENTORNO_n
\end{verbatim}
No hay diferencia en agregar una variable de entorno a un archivo existente o crear un nuevo archivo donde se defina.
\subsection{Instalaci�n para modo salvapantallas}
Si se desea que el esclavo funciona solamente cuando est� en modo salvapantallas, entonces adicionalmente a lo descrito en los pasos anteriores se deben copiar los archivos correspondientes al salvapantallas\footnote{\texttt{sscolmena.scr}} y al programa de seteo de prioridad de procesos\footnote{\texttt{SetPriority.exe}}. El archivo \texttt{sscolmena.scr} debe ser instalado en el sistema operativo\footnote{apretar con el bot�n derecho el archivo y seleccionar la opci�n \emph{Instalar}}. 
\section{Configuraci�n del programa esclavo}
Para ver todas las opciones de que tiene el programa esclavo hay que ejecutar en la l�nea de comando \texttt{colmena help}, entonces aparecer� la siguiente salida.
\begin{verbatim}
Usage :
run                    Run VC Slave
[help]                 Show Help
info                   Show configuration info
setmasterip ip         Set slave master ip
setmasterport port     Set slave master port
setlistenport port     Set slave listen port
setmaxtasks tasks      Set slave max tasks
setpriority priority   Set if slave is always on high priority
setgroup group         Set slave group
setssaverport port     Set ssaver port
\end{verbatim}
\subsection{Asignaci�n de la direcci�n IP del maestro}
La direcci�n IP del maestro, es la direcci�n IP del computador que est� corriento el programa maestro. Por defecto, el programa esclavo tiene asignada la direcci�n \texttt{127.0.0.1 (localhost)} para tal fin la cual no corresponde en la mayor�a de los casos a menos que el programa maestro este corriendo en la misma m�quina que el esclavo. Si se desea cambiar el puerto por defecto se puede hacer con el comando \texttt{master setmasterip ip} donde \texttt{ip} es la nueva direcci�n a asignar.
\begin{ejemplo} Si se desea asignar la ip 152.1.2.3 como direcci�n IP del maestro entonces se debe ingresar por la l�nea de comando \texttt{colmena setmasterip 152.1.2.3}.
\end{ejemplo}
\subsection{Asignaci�n del puerto de comunicaci�n del maestro}
El puerto de comunicaci�n del maestro, es el puerto UDP por el cual el programa maestro establece la comunicaci�n con los esclavos. Por defecto, el programa esclavo tiene asignado el puerto 40000. Si se desea cambiar el puerto por defecto se puede hacer con el comando \texttt{colmena setmasterport port} donde \texttt{port} es el nuevo puerto a asignar.
\begin{ejemplo} Si se desea asignar el puerto 4000 como puerto de comunicaci�n del maestro entonces se debe ingresar por l�nea de comando \texttt{colmena setmasterport 4000}.
\end{ejemplo}
\subsection{Asignaci�n del puerto de comunicaci�n del esclavo}
El puerto de comunicaci�n del maestro es el puerto UDP por el cual el programa esclavo se comunica con el maestro. Si se posee un cortafuegos, hay que cerciorarse que el computador que correr� el programa esclavo tenga tal puerto abierto. Por defecto, el programa esclavo tiene asignado el puerto 40001. Si se desea cambiar el puerto por defecto se puede hacer con el comando \texttt{colmena setlistenport port} donde \texttt{port} es el nuevo puerto a asignar.
\begin{ejemplo} Si se desea asignar el puerto 4001 como puerto de comunicaci�n del maestro entonces se debe ingresar por l�nea de comando \texttt{colmena setlistenport 4001}.
\end{ejemplo}
\subsection{Configuraci�n del n�mero m�ximo de tareas simultaneas}
Como su nombre lo dice, el n�mero m�ximo de tareas simultaneas corresponde al l�mite de tareas que el esclavo puede ejecutar a la vez. Este valor es particularmente �til cuando se dispone de un computador con muchos procesadores disponibles. Por defecto, el n�mero m�ximo de tareas simultaneas es 1. Si se desea cambiar este valor se puede hacer con el comando \texttt{colmena setmaxtasks tasks} donde \texttt{tasks} es el nuevo m�ximo a asignar.
\begin{ejemplo} Si se desea configurar el n�mero m�ximo de tareas simultaneas en 2 tareas simultaneas entonces se debe ingresar por linea de comando \texttt{colmena setmaxtasks 2}.
\end{ejemplo}
\subsection{Asignaci�n de la prioridad del esclavo}
La prioridad del esclavo es un valor booleano que establece si el seclavo estar� en modo dedicado o no.Si el valor es \texttt{0} entonces el esclavo solo recibir� tareas cuando este en modo salvapantallas. Si el valor es \texttt{1} entonces se asume que el esclavo esta en modo dedicado y recibir� tareas siempre. Por defecto, el esclavo esta configurado para que funcione en modo no dedicado. Si se desea cambiar la prioridad del esclavo entonces se puede hacer con el comando \texttt{colmena setpriority priority} donde \texttt{priority} es el valor booleano correspondiente
\begin{ejemplo} Si se desea asignar cambiar a modo dedicado entonces se debe ingresar por l�nea de comando \texttt{colmena setpriority 1}.
\end{ejemplo}
\subsection{Asignaci�n del grupo del esclavo}
Los esclavos son asociados en grupos que determinan las p�ginas que van a desplegar en sus salvapantallas. El grupo de un esclavo es el nombre del grupo a que pertenece es esclavo. Por defecto, el esclavo esta configurado sin ning�n grupo. Si se desea cambiar la grupo del esclavo entonces se puede hacer con el comando \texttt{colmena setgroup group} donde \texttt{group} es el nombre del grupo.
\begin{ejemplo} Si se desea asignar el grupo \texttt{profc} entonces se debe ingresar por l�nea de comando \texttt{colmena setgroup profc}.
\end{ejemplo}
\subsection{Asignaci�n del puerto de comunicaci�n del salvapantallas}
El puerto de comunicaci�n del salvapantallas es el puerto TCP por el cual el esclavo y el salvapantallas se comunican. Por defecto, el esclavo esta configurado con el puerto 18765. Si se desea cambiar el puerto de comunicaci�n del salvapantallas se puede hacer con el comando \texttt{colmena setssaverport port} donde \texttt{port} es el nuevo puerto a asignar.
\begin{ejemplo} Si se desea asignar el puerto 1876 como puerto de comunicaci�n del salvapantallas entonces se debe ingresar por l�nea de comando \texttt{colmena setssaverport 1876}
\end{ejemplo}
\subsection{Ver configuraci�n actual del esclavo}
Con esta opci�n se puede ver los valores de la configuraci�n del esclavo vigentes. Para verlos simplemente hay que ingresar en el comando \texttt{colmena info}. Si la configuraci�n del esclavo solo tiene valores por defecto, entonces la salida mostrada deber�a ser:
\begin{verbatim}
Version                 : 0.02
Master Ip               : 127.0.0.1
Master Port             : 40000
Listen Port             : 40001
Max Tasks               : 1
Always on high priority : 0
Group                   : 
Screensaver Port        : 18765
\end{verbatim}
Si se hubieran reasignado los valores de la configuraci�n con los establecidos en los ejemplos descritos anteriormente, entonces la salida ser�a:
\begin{verbatim}
Version                 : 0.02
Master Ip               : 152.1.2.3
Master Port             : 4000
Listen Port             : 4001
Max Tasks               : 2
Always on high priority : 1
Group                   : profc
Screensaver Port        : 1876
\end{verbatim}
\section{Ejecuci�n del programa esclavo}
Para correr el programa esclavo simplemente hay que ingresar el comando \texttt{colmena run}. El programa esclavo es un proceso permanente, por eso se recomienda ejecutarlo como un proceso de fondo. El programa esclavo se demora un poco en estar disponible para recibir tareas debido a las rutinas de reconocimiento y verificaci�n de aplicaciones disponibles. Cuando el esclavo se encuentre disponible en la salida del maestro aparecer� un mensaje notificando este hecho :
\begin{verbatim}
New slave 1 arrived. Starting registering protocol.
\end{verbatim}
En este punto, el esclavo esta disponible para que el maestro le env�e tareas a ejecutar, en otras palabras, se ha agregado un nodo de c�mputo adicional al cluster virtual.
\section{Instalador autom�tico del esclavo para Windows XP}
Por conveniencia, se cre� un programa instalador autom�tico para Windows XP con InnoSetup\cite{innosetuppage} que tiene el fin de facilitar el proceso de configuraci�n del esclavo. Este instalador deja el programa esclavo como un servicio de Windows permanente que esta preconfigurado para que se conecte a un maestro en espec�fico.
\section{Instalador de paquetes de aplicaciones para Windows XP}
Tambi�n existe un instalador de aplicaciones para el esclavo. Corriendo este instalador agrega autom�ticamente aplicaciones disponibles por el esclavo.
\chapter{Descripci�n de Mensajes y Paquetes de Comunicaci�n}\label{anexo:definicion_mensaje}
Los mensajes o paquetes de comunicaci�n son descritos de la siguiente forma:
\begin{center}
\texttt{| tipo\_1 nombre\_1 | tipo\_2 nombre\_2 | ... | tipo\_n nombre\_n |\\origen>destino}
\end{center}
donde la l�nea superior separa los campos del mensaje con \texttt{|}. Cada campo tiene dos palabras donde la primera describe el tipo de dato que tiene y el segundo es un nombre identificador. En la segunda l�nea se describe la direcci�n en que va el mensaje. Tanto \texttt{origen} como \texttt{destino} son identificadores de entidades. El sentido del caracter \texttt{>} determina cual es la orientaci�n de la comunicaci�n. Eventualmente las dos l�neas pueden encontrarse concatenadas en una sola por conveniencia.

\subsection{Computaci�n de nube}
Como se mencion� anteriormente, cuando se habla de computaci�n de nube no se est� hablando de una tecnolog�a en particular, sino de una soluci�n ofrecida como un servicio. Elegir entre pagar por un servicio o implantar tecnolog�a se puede traducir a la pregunta : \emph{�Es preferible cultivar mi comida o comprarla en el supermercado?}. Dentro del contexto del proyecto esta pregunta ser�a : \emph{�Es preferible usar el cluster virtual o simplemente pagar por el servicio de c�mputo?}.
Para responder a esta pregunta hay que remitirse al objetivo principal del proyecto que es \emph{''aprovechar el poder de c�mputo de los computadores ociosos''}. De esta sentencia se deduce que el cluster virtual est� pensado para aprovechar un potencial no utilizado. Si existe tal potencial entonces �por qu� se tendr�a que pagar adicionalmente por �l?. En resumen, la computaci�n de nube resuelve el problema de necesidad de poder de c�mputo mediante recursos econ�micos, en cambio el sistema de este proyecto resuelve este problema mediante el uso de recursos subutizados, lo que significa una soluci�n m�s �ptima.

