La especificaci�n del trabajo es manejada por el programa cliente y el programa maestro. El programa cliente env�a una especificaci�n al programa maestro que agrega a su lista de trabajos.
Para la implementaci�n de la especificaci�n del trabajo se consideran las siguientes requisitos que debe cumplir:
\begin{itemize}
\item Transferible por la red.
\item Manejar informaci�n contenida en gran cantidad de archivos\footnote{Del orden de los cientos de miles de archivos}.
\end{itemize}
Manejar un gran n�mero de archivos representa un problema. Los sistemas de archivos jer�rquicos no est�n dise�ados para manejar una gran cantidad de archivos en un mismo directorio \cite{filesystem}. Muchas aplicaciones se han visto enfrentadas a este problema. El caso m�s com�n son los servidores de correo electr�nico (QMail, Postfix, etc) que almacenan cada mensaje en un archivo y para evitar listados enormes de archivos, los organizan en una estructura de subdirectorios.
Considerando tambi�n que debe ser transferible por red (y adem�s sobre un protocolo poco fiable como UDP), enviar muchos archivos es poco adecuado y ser�a mejor que un s�lo archivo contuviera todos los datos que comprenden la especificaci�n.

\emph{SQLite3}\cite{sqlite3page} es una biblioteca de software que implementa un motor de base de datos SQL. Las principales caracter�sticas de SQLite3 con respecto a otros motores de base de datos es que no requiere configuraci�n ni de un servidor y adem�s toda la base de datos es contenida dentro de un s�lo archivo. Usando archivos SQLite3 para especificar el trabajo, el usuario tiene la ventaja de poder ver un trabajo como una base de datos relacional accedida por lenguaje SQL y para �mbitos de comunicaci�n, estas bases de datos son tratadas como un simple archivo, que adem�s puede ser comprimido f�cilmente para optimizar su transferencia.
\subsection{Estructura de la base de datos}
En la figura \ref{fig:image14} se muestra un modelo entidad-relaci�n de la especificaci�n del trabajo. En este modelo tenemos tres entidades:
\begin{itemize}
\item \textbf{Tareas :} Esta entidad identifica la tarea a ejecutar. Tiene 4 atributos: Un identificador num�rico �nico, el programa a ejecutar, los argumentos de l�nea de comando de la ejecuci�n de dicho programa y el contenido de la entrada est�ndar. Esta entidad est� asociada con la entidad archivos que comprende a los archivos de entrada necesarios para la ejecuci�n de la tarea; y tambi�n est� asociada con la entidad salidas que son los archivos de resultados a obtener de la ejecuci�n de la tarea.
\item \textbf{Archivos :} Esta entidad representa un archivo de entrada de una o varias tareas. Tiene 3 atributos: Un identificador num�rico �nico, el nombre del archivo y los datos contenidos en el archivo. Un tarea puede tener muchos archivos y un archivo puede estar relacionado con muchas tareas.
\item \textbf{Salidas :} Esta entidad representa los nombres de los archivos de salida de una tarea. Tiene un s�lo atributo y es el nombre del archivo a obtener. Una tarea puede tener muchas salidas.
\end{itemize}
\begin{figure}
\begin{center}
\includegraphics[width=\textwidth]{images/image14.eps}
\end{center}
\caption{Modelo entidad-relaci�n de la especificaci�n de trabajos}
\label{fig:image14}
\end{figure}
Este modelo se traduce en 2 tablas que muestran a continuaci�n:
\subsubsection{Tabla tareas}
\begin{verbatim}
Tareas( id , argumentos , entrada estandar , archivos , salidas )
\end{verbatim}
Atributos :
\begin{itemize}
\item \textbf{id :} Identificador num�rico entero.
\item \textbf{argumentos :} Cadena de caracteres.
\item \textbf{entrada estandar :} Datos binarios.
\item \textbf{archivos:} Lista de identificadores de archivos separados por coma (,).
\item \textbf{salidas:} Lista de nombres de archivos de salida separados por coma (,).	
\end{itemize}
\subsubsection{Tabla archivos}
\begin{verbatim}
Archivos( id , nombre , datos )
\end{verbatim}
Atributos :
\begin{itemize}
\item \textbf{id :} Identificador num�rico entero.
\item \textbf{nombre :} Cadena de caracteres.
\item \textbf{datos :} Datos binarios.
\end{itemize}
\subsection{Especificaci�n SQL}
Las sentencias SQL de creaci�n de las tablas tareas y argumentos son las siguientes:
\begin{verbatim}
CREATE TABLE files ( id INTEGER , name TEXT , data TEXT ) ;
CREATE TABLE tasks ( program TEXT , args TEXT , stdin TEXT ,
                     files TEXT , retrieve TEXT ) ;
\end{verbatim}
La tabla \texttt{files} corresponde a la de \emph{archivos} y la tabla \texttt{tasks} corresponde a la \emph{tareas}. La tabla tasks no posee un campo id ya que en este caso se usa el identificador por defecto que posee una fila de una tabla SQLite3 y que se llama  \texttt{rowid}. Se puede acceder a �l mediante la consulta SQL :
\begin{verbatim}
SELECT rowid FROM tasks ;
\end{verbatim}






