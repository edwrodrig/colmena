El mismo problema mencionado para la especificaci�n de trabajos est� presente con la especificaci�n de resultados. Los resultados son todos los archivos recuperados de la ejecuci�n de todas las tareas de un trabajo. Es claro que el n�mero de archivos de resultados es del mismo orden que la cantidad de archivos de entradas y por ello es que manejarlos simplemente como archivos es  poco deseable.

El m�todo para solucionar este problema es el mismo que se us� para los trabajos. Se usa un archivo SQLite3 con una sola tabla la que se describe a continuaci�n:
\subsubsection{Tabla archivos}
\begin{verbatim}
Archivos( tarea , nombre , datos )
\end{verbatim}
Atributos :
\begin{itemize}
\item \textbf{tarea :} Identificador de la tarea.
\item \textbf{nombre:} Nombre del archivo de resultados.
\item \textbf{datos:} Datos que contiene el archivo.
\end{itemize}
Cada entrada de esta tabla corresponde a un archivo de salida. Con el atributo \emph{nombre} y \emph{tarea} el usuario puede determinar a qu� tarea corresponde el resultado. El c�digo SQL de creaci�n de la tabla es el siguiente:
\begin{verbatim}
CREATE TABLE files ( idtask INTEGER , filename TEXT , data TEXT ) ;
\end{verbatim}