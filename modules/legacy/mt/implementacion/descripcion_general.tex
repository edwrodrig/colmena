El cluster virtual cuenta con cuatro piezas de software: el programa cliente, el programa maestro, el programa esclavo y el salvapantallas. La ejecuci�n de estos programas determinan el rol que est� cumpliendo una m�quina en particular dentro del sistema. Por ejemplo, el maestro es el computador que se encuentre ejecutando el programa maestro como se muestra en la figura \ref{fig:image13}. Estos programas est�n construidos en C++ usando la bibliotecas \emph{framework Qt4}\cite{qtpage}, que tiene la cualidad de permitir portar f�cilmente las aplicaciones a Windows y/o Linux entre otros sistemas operativos.
\begin{figure}
\begin{center}
\includegraphics[width=\textwidth]{images/image13.eps}
\end{center}
\caption{Esquema de roles de la arquitectura del cluster virtual con respecto a los diferentes programas}
\label{fig:image13}
\end{figure}
El programa cliente es una simple aplicaci�n de l�nea de comando que se conecta a trav�s de TCP al programa maestro para ejecutar las funcionalidades que est�n habilitadas para los usuarios. El detalle de la comunicaci�n entre el programa cliente y el programa maestro ser� descrito en el cap�tulo \ref{ref:chapter_comunicacion}.
Los programas maestro y esclavo son m�s complejos que los otros dos, por eso cada uno tiene una secci�n para describir en detalle su implementaci�n.

El caso de la especificaci�n de trabajos, tarea y resultados no es trivial por la condici�n de manejar grandes vol�menes de datos. Por esta raz�n la soluci�n planteada es descrita en las secciones \ref{ref:implementacion_trabajo} y \ref{ref:implementacion_resultado}.
