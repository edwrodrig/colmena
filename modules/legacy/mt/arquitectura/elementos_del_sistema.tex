En esta secci�n se describen los diferentes elementos que son parte del sistema. Tambi�n se definen conceptos necesarios para el desarrollo de las pr�ximas secciones del documento.
\paragraph{Cliente}
Proceso que sirve como medio para que los usuarios interact�en con el cluster virtual. Principalmente esta interacci�n est� enfocada al ingreso de \emph{trabajos} al sistema y su manipulaci�n. Este proceso tambi�n permite la administraci�n de diferentes aspectos del funcionamiento del sistema tales como cuentas de usuarios y administraci�n de esclavos.
\paragraph{Maestro}
Proceso central del sistema de cluster virtual que es el nexo entre los clientes y los esclavos. Posee dos funciones primordiales: recibir y procesar las peticiones de los clientes, y coordinar a los esclavos para la ejecuci�n de los trabajos. 
\paragraph{Esclavo}
Proceso que reside en los nodos destinados al c�mputo. La funci�n de este programa es recibir las tareas desde el maestro y ejecutarlas localmente. Luego de terminadas, el esclavo retorna el resultado de la tarea al maestro.
\paragraph{Tarea}
Es la unidad at�mica de procesamiento dentro del sistema. Se define como una ejecuci�n de un aplicaci�n de l�nea de comando, la que puede comprender argumentos, archivos de entrada y salida, y la entrada y salida est�ndar. Esto es lo que el maestro distribuye a los esclavos para su ejecuci�n. El aprovechamiento del potencial de c�mputo del sistema se debe al paralelismo que permite la divisi�n de un problema en m�ltiples ejecuciones de un programa.
\paragraph{Trabajo}
Es un conjunto l�gico de procesamiento que corresponde a un conjunto de tareas que est�n relacionados por alg�n criterio definido por el usuario. El concepto de trabajo existe dentro del sistema porque dada la naturaleza de �ste, el poder de c�mputo de muchos computadores es solo aprovechable cuando hay un conjunto numeroso de tareas ya que es en este nivel donde ocurre el paralelismo. Los trabajos son construidos por el cliente.
\paragraph{Aplicaci�n}
Es un programa disponible en el cluster virtual para la ejecuci�n de tareas. Cada esclavo tiene un conjunto de aplicaciones disponibles que determina las tareas que puede o no ejecutar.
\paragraph{Usuario}
Un usuario es una entidad registrada en el sistema que tiene un determinado control sobre algunos elementos del sistema. Generalmente un usuario tiene control sobre los trabajos que ingresa, pero existen otros tipos de usuarios que tienen control sobre otras caracter�sticas vinculadas al funcionamiento del sistema como control sobre esclavos, estad�sticas del sistema, control sobre usuarios, control sobre los grupos, etc.
\paragraph{Salvapantallas}
Es un elemento opcional puede estar en las m�quinas esclavas para determinar cu�ndo un esclavo est� disponible o no para su uso en la ejecuci�n de tareas. Como su nombre lo dice, es un salvapantallas que funciona de la manera tradicional, es decir, cuando el computador esta inactivo, el salvapantallas se despliega y notifica al esclavo. Si un esclavo no cuenta con un salvapantallas entonces no puede determinar su estado de disponibilidad autom�ticamente sino debe hacerlo manualmente.
\paragraph{Grupo}
Un grupo es un conjunto de esclavos que comparten un mismo comportamiento de salvapantallas, es decir, muestran un mismo conjunto de p�ginas web. Estas p�ginas pueden ser establecidas y administradas por el usuario due�o del grupo\footnote{Los grupos y su rol dentro del sistema es explicado en la subsecci�n \ref{ref:arquitectura_salvapantallas}}.