El sistema de cluster virtual posee una arquitectura cliente-maestro-esclavo muy similar a la de Q$^2$ADPZ. Un proceso central \emph{maestro} coordina a los procesos \emph{esclavos} que se encuentran en cada computador destinado a brindar poder de c�mputo al sistema. El \emph{cliente} env�a trabajos para procesar al maestro, el que una vez ingresados reparte en los diferentes esclavos que se encuentren a su disposici�n. Para establecer cu�ndo un esclavo se encuentra disponible, el esclavo cuenta con un salvapantallas, donde su activaci�n significa que el esclavo se encuentra disponible para recibir tareas. Alternativamente el esclavo puede configurarse para que est� siempre disponible sin importar el estado del salvapantallas.

Haciendo un paralelo con la arquitectura cliente-servidor, el cliente cumple la misma funci�n, pero el servidor estar�a compuesto por el maestro y los esclavos ya que en su conjunto son los que prestan servicios (de c�mputo) a los usuarios. Es en esta analog�a donde surge la visi�n de cluster virtual, ya que a diferencia de un cluster dedicado (servidor de c�mputo), existe un conjunto de m�quinas no dedicadas que el cliente puede disponer como si fuera un cluster dedicado (cluster virtual). En la figura \ref{fig:image06} se grafica la arquitectura del sistema.
\begin{figure}
\includegraphics[width=\textwidth]{images/image06.eps}
\caption{La arquitectura del sistema de cluster virtual}
\label{fig:image06}
\end{figure}
